% Options for packages loaded elsewhere
\PassOptionsToPackage{unicode}{hyperref}
\PassOptionsToPackage{hyphens}{url}
%
\documentclass[
  ignorenonframetext,
  serif,
  professionalfont,
  usenames,
  dvipsnames,
  aspectratio = 169]{beamer}
\usepackage{pgfpages}
\setbeamertemplate{caption}[numbered]
\setbeamertemplate{caption label separator}{: }
\setbeamercolor{caption name}{fg=normal text.fg}
\beamertemplatenavigationsymbolsempty
% Prevent slide breaks in the middle of a paragraph
\widowpenalties 1 10000
\raggedbottom
\setbeamertemplate{part page}{
  \centering
  \begin{beamercolorbox}[sep=16pt,center]{part title}
    \usebeamerfont{part title}\insertpart\par
  \end{beamercolorbox}
}
\setbeamertemplate{section page}{
  \centering
  \begin{beamercolorbox}[sep=12pt,center]{part title}
    \usebeamerfont{section title}\insertsection\par
  \end{beamercolorbox}
}
\setbeamertemplate{subsection page}{
  \centering
  \begin{beamercolorbox}[sep=8pt,center]{part title}
    \usebeamerfont{subsection title}\insertsubsection\par
  \end{beamercolorbox}
}
\AtBeginPart{
  \frame{\partpage}
}
\AtBeginSection{
  \ifbibliography
  \else
    \frame{\sectionpage}
  \fi
}
\AtBeginSubsection{
  \frame{\subsectionpage}
}
\usepackage{amsmath,amssymb}
\usepackage{lmodern}
\usepackage{iftex}
\ifPDFTeX
  \usepackage[T1]{fontenc}
  \usepackage[utf8]{inputenc}
  \usepackage{textcomp} % provide euro and other symbols
\else % if luatex or xetex
  \usepackage{unicode-math}
  \defaultfontfeatures{Scale=MatchLowercase}
  \defaultfontfeatures[\rmfamily]{Ligatures=TeX,Scale=1}
\fi
% Use upquote if available, for straight quotes in verbatim environments
\IfFileExists{upquote.sty}{\usepackage{upquote}}{}
\IfFileExists{microtype.sty}{% use microtype if available
  \usepackage[]{microtype}
  \UseMicrotypeSet[protrusion]{basicmath} % disable protrusion for tt fonts
}{}
\makeatletter
\@ifundefined{KOMAClassName}{% if non-KOMA class
  \IfFileExists{parskip.sty}{%
    \usepackage{parskip}
  }{% else
    \setlength{\parindent}{0pt}
    \setlength{\parskip}{6pt plus 2pt minus 1pt}}
}{% if KOMA class
  \KOMAoptions{parskip=half}}
\makeatother
\usepackage{xcolor}
\IfFileExists{xurl.sty}{\usepackage{xurl}}{} % add URL line breaks if available
\IfFileExists{bookmark.sty}{\usepackage{bookmark}}{\usepackage{hyperref}}
\hypersetup{
  pdftitle={Multivariate regression models for Twin data},
  pdfauthor={Prof.~Wagner Hugo Bonat},
  hidelinks,
  pdfcreator={LaTeX via pandoc}}
\urlstyle{same} % disable monospaced font for URLs
\newif\ifbibliography
\usepackage{color}
\usepackage{fancyvrb}
\newcommand{\VerbBar}{|}
\newcommand{\VERB}{\Verb[commandchars=\\\{\}]}
\DefineVerbatimEnvironment{Highlighting}{Verbatim}{commandchars=\\\{\}}
% Add ',fontsize=\small' for more characters per line
\newenvironment{Shaded}{}{}
\newcommand{\AlertTok}[1]{\textcolor[rgb]{1.00,0.00,0.00}{#1}}
\newcommand{\AnnotationTok}[1]{\textcolor[rgb]{0.00,0.50,0.00}{#1}}
\newcommand{\AttributeTok}[1]{#1}
\newcommand{\BaseNTok}[1]{#1}
\newcommand{\BuiltInTok}[1]{#1}
\newcommand{\CharTok}[1]{\textcolor[rgb]{0.00,0.50,0.50}{#1}}
\newcommand{\CommentTok}[1]{\textcolor[rgb]{0.00,0.50,0.00}{#1}}
\newcommand{\CommentVarTok}[1]{\textcolor[rgb]{0.00,0.50,0.00}{#1}}
\newcommand{\ConstantTok}[1]{#1}
\newcommand{\ControlFlowTok}[1]{\textcolor[rgb]{0.00,0.00,1.00}{#1}}
\newcommand{\DataTypeTok}[1]{#1}
\newcommand{\DecValTok}[1]{#1}
\newcommand{\DocumentationTok}[1]{\textcolor[rgb]{0.00,0.50,0.00}{#1}}
\newcommand{\ErrorTok}[1]{\textcolor[rgb]{1.00,0.00,0.00}{\textbf{#1}}}
\newcommand{\ExtensionTok}[1]{#1}
\newcommand{\FloatTok}[1]{#1}
\newcommand{\FunctionTok}[1]{#1}
\newcommand{\ImportTok}[1]{#1}
\newcommand{\InformationTok}[1]{\textcolor[rgb]{0.00,0.50,0.00}{#1}}
\newcommand{\KeywordTok}[1]{\textcolor[rgb]{0.00,0.00,1.00}{#1}}
\newcommand{\NormalTok}[1]{#1}
\newcommand{\OperatorTok}[1]{#1}
\newcommand{\OtherTok}[1]{\textcolor[rgb]{1.00,0.25,0.00}{#1}}
\newcommand{\PreprocessorTok}[1]{\textcolor[rgb]{1.00,0.25,0.00}{#1}}
\newcommand{\RegionMarkerTok}[1]{#1}
\newcommand{\SpecialCharTok}[1]{\textcolor[rgb]{0.00,0.50,0.50}{#1}}
\newcommand{\SpecialStringTok}[1]{\textcolor[rgb]{0.00,0.50,0.50}{#1}}
\newcommand{\StringTok}[1]{\textcolor[rgb]{0.00,0.50,0.50}{#1}}
\newcommand{\VariableTok}[1]{#1}
\newcommand{\VerbatimStringTok}[1]{\textcolor[rgb]{0.00,0.50,0.50}{#1}}
\newcommand{\WarningTok}[1]{\textcolor[rgb]{0.00,0.50,0.00}{\textbf{#1}}}
\usepackage{graphicx}
\makeatletter
\def\maxwidth{\ifdim\Gin@nat@width>\linewidth\linewidth\else\Gin@nat@width\fi}
\def\maxheight{\ifdim\Gin@nat@height>\textheight\textheight\else\Gin@nat@height\fi}
\makeatother
% Scale images if necessary, so that they will not overflow the page
% margins by default, and it is still possible to overwrite the defaults
% using explicit options in \includegraphics[width, height, ...]{}
\setkeys{Gin}{width=\maxwidth,height=\maxheight,keepaspectratio}
% Set default figure placement to htbp
\makeatletter
\def\fps@figure{htbp}
\makeatother
\setlength{\emergencystretch}{3em} % prevent overfull lines
\providecommand{\tightlist}{%
  \setlength{\itemsep}{0pt}\setlength{\parskip}{0pt}}
\setcounter{secnumdepth}{-\maxdimen} % remove section numbering
%-----------------------------------------------------------------------
% Logo na capa.

\titlegraphic{
  \vspace{-1em}
  % \includegraphics[height=1.2cm]{config/omega-logo.png}
  \includegraphics[height=1.8cm]{config/logo-duas-cores.png}
  % \includegraphics[height=1.8cm]{config/logo-contorno.png}
  % \includegraphics[height=1.8cm]{config/linkedin-logo-da-pagina.png}
}

% \providecommand{\tightlist}{%
%   \setlength{\itemsep}{0pt}\setlength{\parskip}{0pt}}
% ATTENTION: Redefine o comando acima que é definido pelo template.
% \renewcommand{\tightlist}{}
\renewcommand{\tightlist}{%
  \setlength{\itemsep}{0\baselineskip}
  \setlength{\parskip}{0.25\baselineskip}
}
%-----------------------------------------------------------------------

% \usepackage{xltxtra}
%
% \setbeamerfont{title page}{family=\fontspec{Oswald}}
%\setbeamerfont{title}{family=\fontspec{Oswald-Bold}}
%\setbeamerfont{subtitle}{family=\fontspec{Oswald-Medium}}
%\setbeamerfont{frametitle}{family=\fontspec{Oswald-Bold}}

% \setbeamerfont{title in head/foot}{family=\fontspec{Oswald-Medium}}
% \setbeamerfont{author in head/foot}{family=\fontspec{Oswald-Medium}}
%\setbeamerfont{date in head/foot}{family=\fontspec{Oswald-Light}}
%\setbeamerfont{section in head/foot}{family=\fontspec{Oswald-Light}}
%\setbeamerfont{subsection in head/foot}{family=\fontspec{Oswald-Light}}

\setromanfont{Cardo}
\setsansfont{Cardo}
\setmonofont{inconsolata}

% Font for code. ----------------------------
% \usepackage[scaled=.75]{beramono}
% \usepackage{inconsolata}

% ATTENTION: needs complile with xelatex: `$ xelatex file.tex`
% \usepackage{fontspec}
% \setmonofont{M+ 1m}
% \setmonofont{M+ 1mn}
% \setmonofont{M+ 2m}

%-----------------------------------------------------------------------

% \usepackage{lmodern}
\usepackage{amssymb, amsmath}
\usepackage[makeroom]{cancel}
% \usepackage{ifxetex, ifluatex}
\usepackage{fixltx2e} % provides \textsubscript
%\usepackage[utf8]{inputenc}
\usepackage[shorthands=off, main=brazil]{babel}
\usepackage{graphicx}
\usepackage{xcolor}
\usepackage{setspace}
\usepackage{comment}
\usepackage{icomma}

%-----------------------------------------------------------------------
% Algumas configurações.

\setlength{\parindent}{0pt}
\setlength{\parskip}{6pt plus 2pt minus 1pt}
\setlength{\emergencystretch}{3em}  % prevent overfull lines
% \providecommand{\tightlist}{%
%   \setlength{\itemsep}{0pt}\setlength{\parskip}{0pt}}
\setcounter{secnumdepth}{0}

% Espaço vertical para o ambiente `quote`.
\let\oldquote\quote
\let\oldendquote\endquote
\renewenvironment{quote}{%
  \vspace{1em}\oldquote}{%
  \oldendquote\vspace{1em}}

%-----------------------------------------------------------------------
% Espaçamento entre items para itemize, enumerate e description.

% % itemize.
% \let\itemopen\itemize
% \let\itemclose\enditemize
% \renewenvironment{itemize}{%
%   \itemopen\addtolength{\itemsep}{0.25\baselineskip}}{\itemclose}
%
% % enumerate.
% \let\enumopen\enumerate
% \let\enumclose\endenumerate
% \renewenvironment{enumerate}{%
%   \enumopen\addtolength{\itemsep}{0.25\baselineskip}}{\enumclose}
%
% % description.
% \let\descopen\description
% \let\descclose\enddescription
% \renewenvironment{description}{%
%   \descopen\addtolength{\itemsep}{0.25\baselineskip}}{\descclose}

%-----------------------------------------------------------------------

% \usepackage[hang]{caption}
\usepackage{caption}
\captionsetup{font=footnotesize,
  labelfont={color=mycolor1, footnotesize},
  labelsep=period}

% \providecommand{\tightlist}{%
%   \setlength{\itemsep}{0pt}\setlength{\parskip}{0pt}}

%-----------------------------------------------------------------------
% Imagem de fundo.

\usepackage{tikz}

% Caminho para a imagem de fundo com aspecto 16x9.
% \def\pathtobg{config/wallpapertip_gray-wallpaper_877207.jpg}
\def\pathtobg{config/light-gray-background.jpg}
\ifx\pathtobg\undefined
\else
  \usebackgroundtemplate{
    \tikz[overlay, remember picture]
    \node[opacity=0.5,
          at=(current page.south east),
          anchor=south east,
          inner sep=0pt] {
            \includegraphics[height=\paperheight, width=\paperwidth]{\pathtobg}};
  }
\fi

%-----------------------------------------------------------------------
% Logo.

% \logo{
%   \includegraphics[width=0.6cm]{config/omega-logo.png}
% }

%-----------------------------------------------------------------------
% Definições de esquema de cores.

% Ômega.
% Preto #191824    Azul #102C54    Amarelo #FBCD1D    Magenta #EB0D6A
% http://www.color-hex.com/color-palette/2018
\definecolor{mycolor1}{HTML}{102C54} % Título.
\definecolor{mycolor3}{HTML}{102C54} % Estrutura.
\definecolor{mycolor4}{HTML}{102C54} % Links.
\definecolor{mycolor2}{HTML}{191824} % Texto.
\definecolor{mycolor5}{HTML}{CDCDCD} % Preenchimentos.

\hypersetup{
  colorlinks=true,
  linkcolor=mycolor4,
  urlcolor=mycolor1,
  citecolor=mycolor1
}

%-----------------------------------------------------------------------
% ATTENTION: http://www.cpt.univ-mrs.fr/~masson/latex/Beamer-appearance-cheat-sheet.pdf

\usetheme{Boadilla}
\usecolortheme{default}

% \setbeamersize{text margin left=7mm, text margin right=7mm}
% \setbeamertemplate{frametitle}[default][left, leftskip=3mm]
% \addtobeamertemplate{frametitle}{\vspace{0.5em}}{}

\setbeamertemplate{caption}[numbered]
\setbeamertemplate{section in toc}[sections numbered]
\setbeamertemplate{subsection in toc}[subsections numbered]
\setbeamertemplate{sections/subsections in toc}[ball]{}
\setbeamertemplate{sections in toc}[ball]
\setbeamercolor{section number projected}{bg=mycolor1, fg=white}
\setbeamertemplate{blocks}[rounded]
\setbeamertemplate{navigation symbols}{}
\setbeamertemplate{frametitle continuation}{\gdef\beamer@frametitle{}}
% \setbeamertemplate{frametitle}[default][center]
% \setbeamertemplate{footline}[frame number]

\setbeamertemplate{enumerate items}[default]
\setbeamertemplate{itemize items}{\scriptsize\raise1.25pt\hbox{\donotcoloroutermaths$\blacktriangleright$}}

% Blocos.
% \addtobeamertemplate{block begin}{\vskip -\bigskipamount}{}
% \addtobeamertemplate{block end}{}{\vskip -\bigskipamount}
\addtobeamertemplate{block begin}{\vspace{0.5em}}{}
\addtobeamertemplate{block end}{}{\vspace{0.5em}}

% Rodapé.
\setbeamercolor{title in head/foot}{parent=subsection in head/foot}
\setbeamercolor{author in head/foot}{bg=mycolor4, fg=white}
\setbeamercolor{date in head/foot}{parent=subsection in head/foot, fg=mycolor3}

% Cabeçalho.
\setbeamercolor{section in head/foot}{bg=mycolor2, fg=mycolor4}
\setbeamercolor{subsection in head/foot}{bg=mycolor2, fg=white}

\setbeamercolor{title}{fg=mycolor1}       % Título dos slides.
\setbeamercolor{titlelike}{fg=title}
\setbeamercolor{subtitle}{fg=mycolor2}    % Subtítulo.
\setbeamercolor{institute in head/foot}{parent=palette primary} % Instituição.
\setbeamercolor{frametitle}{fg=mycolor1}  % De quadro.
\setbeamercolor{structure}{fg=mycolor3}   % Listas e rodapé.
\setbeamercolor{item projected}{bg=mycolor2}
\setbeamercolor{block title}{bg=mycolor5, fg=mycolor2}
\setbeamercolor{normal text}{fg=mycolor2} % Texto.
\setbeamercolor{caption name}{fg=normal text.fg}
% \setbeamercolor{footlinecolor}{fg=mycolor2, bg=mycolor5}
% \setbeamercolor{section in head/foot}{fg=mycolor2, bg=mycolor5}
\setbeamercolor{author in head/foot}{fg=white, bg=mycolor1}
\setbeamercolor{section in foot}{fg=mycolor4, bg=mycolor5}
\setbeamercolor{date in foot}{fg=mycolor4, bg=mycolor5}
\setbeamercolor{block title}{fg=white, bg=mycolor1}
\setbeamercolor{block body}{fg=black, bg=white!80!gray}
\setbeamercolor{block body}{fg=black, bg=white!80!gray}

% To remove empty brackets of \institution.
\makeatletter
\setbeamertemplate{footline}{
  \leavevmode%
  \hbox{%
    \begin{beamercolorbox}[
      wd=0.3\paperwidth, ht=2.25ex, dp=1ex, right]{author in head/foot}%
      \usebeamerfont{author in head/foot}\insertshortauthor{}\hspace*{1ex}
    \end{beamercolorbox}%
    \begin{beamercolorbox}[
      wd=0.6\paperwidth, ht=2.25ex, dp=1ex, left]{section in foot}%
      \usebeamerfont{title in head/foot}\hspace*{1ex}\insertshorttitle{}
      % \usebeamerfont{title in head/foot}\hspace*{1ex}\insertframetitle{}
    \end{beamercolorbox}%
    \begin{beamercolorbox}[
      wd=0.1\paperwidth, ht=2.25ex, dp=1ex, right]{date in foot}%
      \insertframenumber{}\hspace*{2ex}
    \end{beamercolorbox}
  }%
  \vskip0pt%
}
\makeatother

%-----------------------------------------------------------------------

% \usepackage{hyphenat}
\usepackage{changepage}

% Slide para o título das seções.
\AtBeginSection[]{
  \begin{frame}
    % \vfill
    \vspace{4cm}
    % \centering
    % \begin{beamercolorbox}[sep = 8pt, center, shadow = true, rounded = true]{title}
    \begin{beamercolorbox}{title}
      \begin{columns}
        \column{0.7\linewidth}
        {\LARGE\textbf \insertsectionhead}
      \end{columns}
    \end{beamercolorbox}
    \vfill
  \end{frame}
}

%-----------------------------------------------------------------------
%---- preamble-chunk.tex -----------------------------------------------

% Knitr.

% ATTENTION: this needs `\usepackage{xcolor}'.
\definecolor{color_line}{HTML}{333333}
\definecolor{color_back}{HTML}{DDDDDD}
% \definecolor{color_back}{HTML}{FF0000}

% ATTENTION: usa o fancyvrb.
% https://ctan.math.illinois.edu/macros/latex/contrib/fancyvrb/doc/fancyvrb-doc.pdf
% R input.
\usepackage{tcolorbox}
\ifcsmacro{Highlighting}{
  % Statment if it exists. ------------------
  \DefineVerbatimEnvironment{Highlighting}{Verbatim}{
    % frame=lines,     % Linha superior e inferior.
    % framesep=1ex,    % Distância da linha para o texto.
    % framerule=0.5pt, % Espessura da linha.
    % rulecolor=\color{color_line},
    % numbers=right,
    fontsize=\footnotesize, % Tamanho da fonte.
    baselinestretch=0.9,   % Espaçamento entre linhas.
    commandchars=\\\{\}}
  % Margens do ambiente `Shaded'.
  % \fvset{listparameters={\setlength{\topsep}{-1em}}}
  % \renewenvironment{Shaded}{\vspace{-1ex}}{\vspace{-2ex}}
  \renewenvironment{Shaded}{
    \vspace{2pt}
    \begin{tcolorbox}[
      boxrule=0pt,      % Espessura do contorno.
      colframe=gray!10, % Cor do contorno.
      colback=gray!10,  % Cor de fundo da caixa.
      % arc=1em,          % Raio para contornos arredondados.
      sharp corners,
      % boxsep=0.5em,     % Margem interna.
      left=3pt, right=3pt, top=3pt, bottom=3pt, % Margens internas.
      % grow to left by=0mm,
      grow to right by=6pt,
      ]
    }{
    \end{tcolorbox}
    \vspace{-3pt}
    }
  }{
  % Statment if it not exists. --------------
}

% R output e todo `verbatim'.
\makeatletter
\def\verbatim@font{\linespread{0.9}\normalfont\ttfamily\footnotesize}
\makeatother

% Cor de fundo e margens do `verbatim'.
\let\oldv\verbatim
\let\oldendv\endverbatim

% \def\verbatim{%
%   \par\setbox0\vbox\bgroup % Abre grupo.
%   \vspace{-5px}            % Reduz margem superior.
%   \oldv                    % Chama abertura do verbatim.
% }
% \def\endverbatim{%
%   \oldendv                 % Chama encerramento do verbatim.
%   % \vspace{0cm}           % Controla margem inferior.
%   \egroup%\fboxsep5px      % Fecha grupo.
%   \noindent{\colorbox{color_back}{\usebox0}}\par
% }

%-----------------------------------------------------------------------
%---- preamble-commands.tex --------------------------------------------

% Para fazer texto em duas colunas.
\newcommand{\mytwocolumns}[4]{
  % #1: Line width fraction for the left column , e.g. 0.5.
  % #2: Line width fraction for the right column.
  % #3: Content for the left column.
  % #4: Content for the right column.
  \begin{columns}[c]
    \begin{column}{#1\linewidth} %----------- left.
      #3
    \end{column} %--------------------------- left.
    \begin{column}{#2\linewidth} %----------- right.
      #4
    \end{column} %--------------------------- right.
  \end{columns}
}

%-----------------------------------------------------------------------
% Para fazer duas colunas no Rmd.

% Center vertical align.
\def\beginAHalfColumn{\begin{minipage}{0.49\textwidth}}%
\def\beginAlmostHalfColumn{\begin{minipage}{0.45\textwidth}}%
\def\beginAQuarterColumn{\begin{minipage}{0.23\textwidth}}%
\def\beginThreeQuartersColumn{\begin{minipage}{0.72\textwidth}}%
\def\beginAThirdColumn{\begin{minipage}{0.31\textwidth}}%
\def\beginTwoThirdsColumn{\begin{minipage}{0.64\textwidth}}%
\def\endColumns{\end{minipage}}%

% Top vertical align.
\def\beginAHalfColumnT{\begin{minipage}[t]{0.49\textwidth}}%
\def\beginAlmostHalfColumnT{\begin{minipage}[t]{0.45\textwidth}}%
\def\beginAQuarterColumnT{\begin{minipage}[t]{0.23\textwidth}}%
\def\beginThreeQuartersColumnT{\begin{minipage}[t]{0.72\textwidth}}%
\def\beginAThirdColumnT{\begin{minipage}[t]{0.31\textwidth}}%
\def\beginTwoThirdsColumnT{\begin{minipage}[t]{0.64\textwidth}}%

%---------------------------------------------------------------------
% Ambientes para frases como e sem imagem.

\newcommand{\myquote}[3]{
  % #1: caminho para a imagem.
  % #2: a frase/quotation.
  % #3: o autor.
  \begin{center}
    \begin{minipage}[c]{0.19\linewidth}
      \begin{center}
        \includegraphics[height=2.5cm]{#1}
      \end{center}
    \end{minipage}
    \begin{minipage}[c]{0.7\linewidth}
      \begin{flushright}
        \textit{#2}
        \vspace{1ex}

        -- #3
      \end{flushright}
    \end{minipage}
  \end{center}
}

\newcommand{\myphrase}[2]{
  % #1: a frase/quotation.
  % #2: o autor.
  \begin{center}
    \begin{minipage}[c]{0.19\linewidth}
    \end{minipage}
    \begin{minipage}[c]{0.7\linewidth}
      \begin{flushright}
        \textit{#1}
        \vspace{1ex}

        -- #2
      \end{flushright}
    \end{minipage}
  \end{center}
}

%-----------------------------------------------------------------------
% Comandos para texto em destaque.

% \newcommand{\hi}[1]{%
%   \textcolor{ubuntu_orange}{#1}\xspace
% }

\usepackage{xspace}

% URLs com letra miuda.
\newcommand{\myurl}[1]{%
  {\tiny \url{#1}}\xspace
}

% Botões.
\newcommand{\btn}[1]{%
  \beamergotobutton{#1}\xspace
}

% Texto grande centralizado.
\newcommand{\centertitle}[1]{%
  \begin{center}
    {\LARGE \bfseries \hi{#1}}
  \end{center}
}

%-----------------------------------------------------------------------
\ifLuaTeX
  \usepackage{selnolig}  % disable illegal ligatures
\fi

\title{Multivariate regression models for Twin data}
\subtitle{Analysis of Twin Data in Health Science · Session IV}
\author{Prof.~Wagner Hugo Bonat}
\date{}
\institute{Ômega Data Science \textbar{} Online School of Data Science}

\begin{document}
\frame{\titlepage}

\hypertarget{motivation}{%
\section{Motivation}\label{motivation}}

\begin{frame}{Heredity and variation}
\protect\hypertarget{heredity-and-variation}{}
\beginAHalfColumn

\begin{itemize}
\tightlist
\item
  Genetic epidemiology is impelled by three basic questions:

  \begin{enumerate}
  \tightlist
  \item
    Why isn't everyone the same?
  \item
    Why are children like their parents?
  \item
    Why aren't children from the same parents all alike?
  \end{enumerate}
\item
  \textbf{Main goal: Isolate/Separate sources of variation!}
\end{itemize}

\endColumns
\beginAHalfColumn

\begin{itemize}
\tightlist
\item
  Variation is everywhere!
\end{itemize}

\begin{figure}

{\centering \includegraphics[width=1\linewidth]{Slides_files/figure-beamer/unnamed-chunk-2-1} 

}

\caption{Histogram of height and weight.}\label{fig:unnamed-chunk-2}
\end{figure}

\endColumns
\end{frame}

\begin{frame}[fragile]{Motivating dataset: Anthropometric measures}
\protect\hypertarget{motivating-dataset-anthropometric-measures}{}
\beginAHalfColumn

\begin{itemize}
\tightlist
\item
  Anthropometric measurements (weight and height).
\item
  861 twin pairs: 327 DZ (dizygotic) and 534 MZ (monozygotic).
\item
  Bivariate continuous traits.
\item
  Covariates: age and group.
\item
  Available as an example in the OpenMx package (Neale, et al., 2016).
\item
  Easy access from the \texttt{mglm4twin} package.
\end{itemize}

\endColumns
\beginAHalfColumn

\begin{figure}

{\centering \includegraphics[width=0.8\linewidth]{./img/balanca} 

}

\caption{Photo by Pixabay.}\label{fig:unnamed-chunk-3}
\end{figure}

\endColumns
\end{frame}

\begin{frame}[fragile]{Motivating dataset: Anthropometric measures}
\protect\hypertarget{motivating-dataset-anthropometric-measures-1}{}
\begin{itemize}
\tightlist
\item
  The dataset
\end{itemize}

\begin{Shaded}
\begin{Highlighting}[]
\FunctionTok{library}\NormalTok{(mglm4twin)}
\FunctionTok{data}\NormalTok{(anthro)}
\FunctionTok{glimpse}\NormalTok{(anthro)}
\end{Highlighting}
\end{Shaded}

\begin{verbatim}
## Rows: 1,722
## Columns: 6
## $ weight    <int> 62, 55, 66, 73, 51, 44, 52, 57, 54, 54, 58, 57, ~
## $ height    <dbl> 1.6499, 1.6299, 1.6599, 1.7000, 1.7300, 1.5698, ~
## $ age       <int> 24, 24, 20, 20, 20, 20, 26, 26, 20, 20, 22, 22, ~
## $ Group     <fct> DZ, DZ, DZ, DZ, DZ, DZ, DZ, DZ, DZ, DZ, DZ, DZ, ~
## $ Twin      <int> 535, 535, 536, 536, 537, 537, 538, 538, 539, 539~
## $ Twin_pair <int> 1, 2, 1, 2, 1, 2, 1, 2, 1, 2, 1, 2, 1, 2, 1, 2, ~
\end{verbatim}
\end{frame}

\begin{frame}{Graphing and Quantifying Familial Resemblance}
\protect\hypertarget{graphing-and-quantifying-familial-resemblance}{}
\begin{figure}

{\centering \includegraphics[width=0.8\linewidth]{Slides_files/figure-beamer/unnamed-chunk-5-1} 

}

\caption{Dispersion diagram by zygosity · Trait weight.}\label{fig:unnamed-chunk-5}
\end{figure}
\end{frame}

\begin{frame}{Multiple traits}
\protect\hypertarget{multiple-traits}{}
\begin{figure}

{\centering \includegraphics[width=0.8\linewidth]{Slides_files/figure-beamer/unnamed-chunk-6-1} 

}

\caption{Dispersion diagram by zygosity · Weight vs Height.}\label{fig:unnamed-chunk-6}
\end{figure}
\end{frame}

\begin{frame}{Building and Fitting Models}
\protect\hypertarget{building-and-fitting-models}{}
\begin{figure}

{\centering \includegraphics[width=0.75\linewidth]{./img/model} 

}

\caption{Diagram of the interrelationship between theory, model and empirical observation. Adapted from Neale and Maes (1992).}\label{fig:unnamed-chunk-7}
\end{figure}
\end{frame}

\begin{frame}{Challenges for model-building in Twin data analyses}
\protect\hypertarget{challenges-for-model-building-in-twin-data-analyses}{}
\beginAHalfColumn

\begin{itemize}
\tightlist
\item
  Decompose sources of variation

  \begin{enumerate}
  \tightlist
  \item
    Genetic Effects.
  \item
    Environmental Effects.
  \item
    Genotype-Environment Interaction.
  \end{enumerate}
\item
  Traits types

  \begin{enumerate}
  \tightlist
  \item
    Binary and binomial data.
  \item
    Bounded data and continuous proportions.
  \item
    Under-, equi- and over-dispersed count data.
  \item
    Semi-continuous data (continuous \(+\) mass at zero).
  \item
    Symmetric and assymetric continuous data.
  \end{enumerate}
\item
  Multiple traits of mixed types.
\end{itemize}

\endColumns
\beginAHalfColumn

\begin{figure}

{\centering \includegraphics[width=0.8\linewidth]{./img/blocks} 

}

\caption{Photo by Magda Ehlers from Pexels.}\label{fig:unnamed-chunk-8}
\end{figure}

\endColumns
\end{frame}

\begin{frame}{Importance and statistical approaches}
\protect\hypertarget{importance-and-statistical-approaches}{}
\begin{itemize}
\tightlist
\item
  Multivariate twin and family studies are important tools to:

  \begin{enumerate}
  \tightlist
  \item
    Determine traits inheritance;
  \item
    Determine the influence of genetic and environmental effects on
    traits.
  \end{enumerate}
\item
  Statistical challenge:

  \begin{enumerate}
  \tightlist
  \item
    Model the covariance structure to take into account the genetic and
    environmental structures induced by the twin and family designs.
  \end{enumerate}
\item
  Orthodox approaches:

  \begin{enumerate}
  \tightlist
  \item
    Structural equation modelling (SEM); 2. Linear mixed models (LMM).
  \end{enumerate}
\item
  Main limitations of SEM and LMM:

  \begin{enumerate}
  \tightlist
  \item
    Both deal only with Gaussian (symmetric) data;
  \item
    Standard computational implementations are difficult to adapt for
    the analysis of twin and family data.
  \end{enumerate}
\end{itemize}
\end{frame}

\hypertarget{multivariate-generalized-linear-models}{%
\section{Multivariate generalized linear
models}\label{multivariate-generalized-linear-models}}

\begin{frame}[fragile]{Multivariate generalized linear models (mglm):
What is it?}
\protect\hypertarget{multivariate-generalized-linear-models-mglm-what-is-it}{}
\begin{itemize}
\tightlist
\item
  Flexible statistical modelling framework to deal with multivariate
  traits.
\item
  Tailored for twin and family data by Bonat and Hjelmborg (2022).
\item
  The mglm approach deals with:

  \begin{enumerate}
  \tightlist
  \item
    Binary and binomial data;
  \item
    Bounded data and continuous proportions;
  \item
    Under-, equi- and over-dispersed count data.
  \item
    Semi-continuous data (continuous \(+\) mass at zero);
  \item
    Symmetric and assymetric continuous data.
  \item
    Combination of all the previous mentioned data.
  \end{enumerate}
\item
  Estimation and inference based on estimating functions (Bonat and
  Jorgense, 2016).
\item
  Computational implementation available through the \texttt{mglm4twin}
  package.
\end{itemize}
\end{frame}

\begin{frame}{Multivariate generalized linear models (mglm):
Non-standard features}
\protect\hypertarget{multivariate-generalized-linear-models-mglm-non-standard-features}{}
\beginAHalfColumn

\begin{itemize}
\tightlist
\item
  Extend standard measures of genetic studies such as:

  \begin{enumerate}
  \tightlist
  \item
    Bivariate heritability, environmentability and common
    environmentability;
  \item
    Genetic, environmental and phenotypic correlations,
  \end{enumerate}
\end{itemize}

to non-Gaussian traits.

\begin{itemize}
\tightlist
\item
  Provide a flexible framework for modelling the dispersion parameters
  as functions of potential covariates.
\item
  Provide software implementation in \texttt{R}.
\end{itemize}

\endColumns
\beginAHalfColumn

\begin{figure}

{\centering \includegraphics[width=0.8\linewidth]{./img/Paper} 

}

\end{figure}

\endColumns
\end{frame}

\hypertarget{multivariate-generalized-linear-models-for-twin-data}{%
\section{Multivariate generalized linear models for twin
data}\label{multivariate-generalized-linear-models-for-twin-data}}

\begin{frame}{Generalized linear models for twin data}
\protect\hypertarget{generalized-linear-models-for-twin-data}{}
\begin{itemize}
\tightlist
\item
  Let \(Y_{i}\) be a \(2 \times 1\) random vector of the i\(th\) twin
  pair for \(i = 1, \ldots, n\).
\item
  Let \(\mathbf{x}_{i} = (x_{i1}, \ldots, x_{ik})^{\top}\) denote a
  \(2 \times k\) design matrix.
\item
  Let \(\boldsymbol{\beta}\) be a \(k \times 1\) parameter vector.
\item
  Consider \((y_{i},\mathbf{x}_{i})\), where \(y_{i}'s\) are iid
  realizations of \(Y_{i}\) according to an \textbf{unspecified}
  bivariate distribution, whose expectation and variance are given by
  \begin{eqnarray}
  \label{modelGLM}
  \mathrm{E}(Y_i) &=& \mu_i = g^{-1}(\mathbf{x}_{i}^{\top}\beta)  \nonumber \\
  \mathrm{var}(Y_i) &=& \Sigma_i = \mathrm{V}(\mu_i;p)^{\frac{1}{2}}\Omega\mathrm{V}(\mu_i;p)^{\frac{1}{2}}.
  \end{eqnarray}
\item
  \(g\) some suitable link function.
\item
  \(\mathrm{V}(\mu_i;p) = \mathrm{diag}(\vartheta(\mu_i;p))\), where
  \(\vartheta(\mu_i;p)\) describes the mean and variance relation and
  \(p\) is the power parameter (to be estimated).
\item
  \(\Omega\) is a \(2 \times 2\) dispersion matrix.
\end{itemize}
\end{frame}

\begin{frame}{Generalized linear models for twin data}
\protect\hypertarget{generalized-linear-models-for-twin-data-1}{}
\begin{itemize}
\tightlist
\item
  The models decompose the covariance matrix into two components.
  \[ \mathrm{var}(Y_i) = \Sigma_i = \mathrm{V}(\mu_i;p)^{\frac{1}{2}}\Omega\mathrm{V}(\mu_i;p)^{\frac{1}{2}}\]
\item
  \(\mathrm{V}(\mu_i;p)\) deals with non-Gaussianity.
\item
  Variance/dispersion functions

  \begin{enumerate}
  \tightlist
  \item
    \(\vartheta(\mu;p) = \mu^p\) characterizes the Tweedie distribution
    deals with continuous and semi-continuous data. Gaussian \((p=0)\),
    Gamma \((p=2)\) and inverse Gaussian \((p=3)\).
  \item
    \(\vartheta(\mu;p) = \mu + \tau \mu^p\) characterizes the
    Poisson-Tweedie distribution deals with count data. Neyman-type A
    \((p=1)\), negative binomial \((p=2)\) and PIG \((p=3)\).
  \item
    \(\vartheta(\mu;p) = \mu^p (1- \mu)^p\) generalization of binomial
    variance function deals with binary, binomial and bounded data.
  \end{enumerate}
\item
  \(p\) is an index that identifies the distribution.
\item
  In practice, we estimate \(p\) which works as an automatic model
  selection.
\end{itemize}
\end{frame}

\begin{frame}{Generalized linear models for twin data}
\protect\hypertarget{generalized-linear-models-for-twin-data-2}{}
\begin{itemize}
\tightlist
\item
  \(\Omega\) models the dependence between twin pair.
\item
  Polygenic ACDE model has the components \begin{equation*}
  \label{components}
  \mathrm{A} = \begin{bmatrix}
  1 & a\\ 
  a & 1
  \end{bmatrix}, \quad
  \mathrm{C} = \begin{bmatrix}
  1 & 1\\ 
  1 & 1
  \end{bmatrix}, \quad
  \mathrm{D} = \begin{bmatrix}
  1 & d\\ 
  d & 1
  \end{bmatrix}\quad \text{and} \quad
  \mathrm{E} \begin{bmatrix}
  1 & 0\\ 
  0 & 1
  \end{bmatrix}.
  \end{equation*}
\item
  Dispersion matrix is modelled by \begin{equation}
  \label{linearCovariance}
  \Omega = \tau_A \mathrm{A} + \tau_C \mathrm{C} + \tau_D \mathrm{D} + \tau_E \mathrm{E},
  \end{equation} where \(a=1\) and \(d = 1\) for MZ twins and
  \(a=\frac{1}{2}\) and \(d = \frac{1}{4}\) for DZ twins.
\item
  Plugging Eq.(\ref{linearCovariance}) in Eq.(\ref{modelGLM}), we have a
  flexible class of models to deal with twin data.
\item
  But, still only one trait.
\end{itemize}
\end{frame}

\begin{frame}{Multivariate GLMs for twin data}
\protect\hypertarget{multivariate-glms-for-twin-data}{}
\begin{itemize}
\tightlist
\item
  Let \(\mathbf{Y}_{ir}\) be the \(2 \times 1\) response vector of the
  r\(th\) trait for \(r = 1, \ldots, R\).
\item
  Let \(\mathbf{x}_{ir} = (x_{ir1}, \ldots, x_{irk})^{\top}\) be the
  \(2 \times k_r\) design matrix.
\item
  Let \(\boldsymbol{\beta}_r\) be the \(k_r \times 1\) parameter
  vectors.
\item
  Let \(\mathbf{Y}_i = (Y_{i1}^{\top}, \ldots, Y_{iR}^{\top}){^\top}\)
  denote the \(2R \times 1\) stacked vector of response variables.
\item
  Multivariate GLMs for twin data \begin{eqnarray}
  \label{modelMGLM}
  \mathrm{E}(\mathbf{Y}_i) &=& \boldsymbol{\mu}_{i} = (g_1^{-1}(\mathbf{x}_{i1}^{\top} \boldsymbol{\beta}_1), \ldots, g_{R}^{-1}(\mathbf{x}_{iR}^{\top} \boldsymbol{\beta}_R))  \nonumber \\
  \mathrm{var}(\mathbf{Y}_i) &=& \boldsymbol{\Sigma}_i = \mathrm{V}(\boldsymbol{\mu}_i;\boldsymbol{p})^{\frac{1}{2}}\boldsymbol{\Omega}\mathrm{V}(\boldsymbol{\mu}_i;\boldsymbol{p})^{\frac{1}{2}}.
  \end{eqnarray}
\item
  \(\mathrm{V}(\boldsymbol{\mu}_i; \boldsymbol{p}) = \mathrm{diag}(\vartheta_1(\mu_1;p_1), \ldots, \vartheta_R(\mu_R;p_R))\),
\item
  \(\boldsymbol{p} = (p_1, \ldots, p_R)\) is an \(R \times 1\) vector of
  power parameters.
\item
  \(\boldsymbol{\Omega}\) is a \(2R \times 2R\) dispersion matrix.
\end{itemize}
\end{frame}

\begin{frame}{Multivariate GLMs for twin data}
\protect\hypertarget{multivariate-glms-for-twin-data-1}{}
\begin{itemize}
\tightlist
\item
  Specification of \(\boldsymbol{\Omega}\) is crucial.
\item
  Let \(\boldsymbol{\nabla}_{r r^{\prime}}\) denote an \(R \times R\)
  matrix, whose entries \(r = r^{\prime}\) and \(r^{\prime} = r\) are
  equal to \(1\) and \(0\) elsewhere, for \(r = 1, \ldots, R\) and
  \(r^{\prime} \leq r\). \begin{eqnarray}
  \label{multACDE}
  \boldsymbol{\Omega} &=& 
  \tau_{A_{rr\prime}} \left\{ \boldsymbol{\nabla}_{rr\prime} \otimes \mathrm{A} \right\} +  
  \tau_{C_{rr\prime}} \left\{ \boldsymbol{\nabla}_{rr\prime} \otimes \mathrm{C} \right\} \nonumber \\ &+& 
  \tau_{D_{rr\prime}} \left\{ \boldsymbol{\nabla}_{rr\prime} \otimes \mathrm{D} \right\} +  
  \tau_{E_{rr\prime}} \left\{ \boldsymbol{\nabla}_{rr\prime} \otimes \mathrm{E} \right\},
  \end{eqnarray} where
  \(\tau_{A_{rr\prime}}, \tau_{C_{rr\prime}}, \tau_{D_{rr\prime}}\) and
  \(\tau_{E_{rr\prime}}\) are dispersion parameters associated with the
  additive genetic, common environment, dominance genetic and unique
  environment effects.
\end{itemize}
\end{frame}

\begin{frame}{Dispersion matrix}
\protect\hypertarget{dispersion-matrix}{}
\begin{itemize}
\tightlist
\item
  Bivariate case \small\{ \begin{eqnarray*}
  \boldsymbol{\Omega} = &\tau_{A_{11}} & \nonumber
  \begin{bmatrix}
  \mathrm{A} & \mathrm{0} \\ 
  \mathrm{0} & \mathrm{0}
  \end{bmatrix} + \tau_{A_{22}}
  \begin{bmatrix}
  \mathrm{0} & \mathrm{0} \\ 
  \mathrm{0} & \mathrm{A}
  \end{bmatrix} + \tau_{A_{12}}
  \begin{bmatrix}
  \mathrm{0} & \mathrm{A} \\ 
  \mathrm{A} & \mathrm{0}
  \end{bmatrix} + \\ \nonumber
  & \tau_{C_{11}} &
  \begin{bmatrix}
  \mathrm{C} & \mathrm{0} \\ 
  \mathrm{0} & \mathrm{0}
  \end{bmatrix} + \tau_{C_{22}}
  \begin{bmatrix}
  \mathrm{0} & \mathrm{0} \\ 
  \mathrm{0} & \mathrm{C}
  \end{bmatrix} + \tau_{C_{12}}
  \begin{bmatrix}
  \mathrm{0} & \mathrm{C} \\ 
  \mathrm{C} & \mathrm{0}
  \end{bmatrix} + \\
  &\tau_{E_{11}}& 
  \begin{bmatrix}
  \mathrm{E} & \mathrm{0} \\ 
  \mathrm{0} & \mathrm{0}
  \end{bmatrix} + \tau_{E_{22}}
  \begin{bmatrix}
  \mathrm{0} & \mathrm{0} \\ 
  \mathrm{0} & \mathrm{E}
  \end{bmatrix} + \tau_{E_{12}}
  \begin{bmatrix}
  \mathrm{0} & \mathrm{E} \\ 
  \mathrm{E} & \mathrm{0}
  \end{bmatrix}.
  \end{eqnarray*} \}
\item
  Note: ACDE model is unidentifiable.
\end{itemize}
\end{frame}

\begin{frame}{Measures of interest}
\protect\hypertarget{measures-of-interest}{}
\begin{itemize}
\tightlist
\item
  Broad sense bivariate heritability, common environmentality and
  environmentality: \begin{eqnarray*}
  h_{rr^{\prime}} &=& \frac{\tau_{A_{rr^{\prime}}} + \tau_{D_{rr^{\prime}}} } {\tau_{A_{rr^{\prime}}} + \tau_{C_{rr^{\prime}}} + \tau_{E_{rr^{\prime}}} }, \\
  c_{rr^{\prime}} &=& \frac{\tau_{C_{rr^{\prime}}} } {\tau_{A_{rr^{\prime}}} + \tau_{C_{rr^{\prime}}} + \tau_{E_{rr^{\prime}}} } \quad \text{and} \quad \\
  e_{rr^{\prime}} &=& \frac{\tau_{E_{rr^{\prime}}} } {\tau_{A_{rr^{\prime}}} + \tau_{C_{rr^{\prime}}} + \tau_{E_{rr^{\prime}}} }.
  \end{eqnarray*}
\end{itemize}
\end{frame}

\begin{frame}{Measures of interest}
\protect\hypertarget{measures-of-interest-1}{}
\begin{itemize}
\item
  Genetic, common environmental and environmental correlations:
  \begin{eqnarray*}
  r_{G_{rr^{\prime}}} &=& \frac{ \tau_{A_{rr^{\prime}}} + \tau_{D_{rr^{\prime}}} }{ \sqrt{\tau_{A_{rr}} + \tau_{D_{rr}}} \sqrt{\tau_{A_{r^{\prime}r^{\prime}}} + \tau_{D_{r^{\prime}r^{\prime}}} } },  \\
  r_{C_{rr^{\prime}}} &=& \frac{ \tau_{C_{rr^{\prime}}} }{ \sqrt{\tau_{C_{rr}}} \sqrt{\tau_{C_{r^{\prime}r^{\prime}}} } } \quad \text{and} \quad
  r_{E_{rr^{\prime}}} = \frac{ \tau_{E_{rr^{\prime}}} }{ \sqrt{\tau_{E_{rr}}} \sqrt{\tau_{E_{r^{\prime}r^{\prime}}} } }.
  \end{eqnarray*}
\item
  Phenotypic correlation \begin{equation*}
  r_{P_{rr^{\prime}}} = \frac{\tau_{Prr^{\prime}}}{\sqrt{ \tau_{Prr}}\sqrt{ \tau_{Pr^{\prime}r^{\prime}}} },
  \end{equation*} where
  \(\tau_{Prr^{\prime}} = \tau_{Arr^{\prime}} + \tau_{C{rr^{\prime}}} + \tau_{D{rr^{\prime}}} + \tau_{E{rr^{\prime}}}.\)
\end{itemize}
\end{frame}

\begin{frame}{Modelling the dispersion parameters}
\protect\hypertarget{modelling-the-dispersion-parameters}{}
\begin{itemize}
\tightlist
\item
  Interest to model the component \(\tau_A\) in a regression model
  fashion.
\item
  Let \(\boldsymbol{z}_i\) be a \((2 \times q)\) design matrix.
  \begin{equation*}
  \boldsymbol{z}_i = \begin{bmatrix}
  1 & z_{i11} & \ldots & z_{i1q} \\ 
  1 & z_{i21} & \ldots & z_{i2q}
  \end{bmatrix}.
  \end{equation*}
\item
  Let
  \(\boldsymbol{\tau_A} = (\tau_A(0), \tau_A(1), \ldots, \tau_A(q))\)
  denote the \((q \times 1)\) vector of dispersion parameters.
\item
  Dispersion components associated to the additive genetic effect
  \begin{equation}
  \label{regdisp}
  \tau_{A(0)}
  \begin{bmatrix}
  1\\ 
  1
  \end{bmatrix} \circ \mathrm{A} + 
  \tau_{A(1)} 
  \begin{bmatrix}
  z_{i11}\\ 
  z_{i21}
  \end{bmatrix} \circ \mathrm{A} + \ldots 
  \tau_{A(q)} 
  \begin{bmatrix}
  z_{i1q}\\ 
  z_{i2q}
  \end{bmatrix} \circ \mathrm{A},
  \end{equation} where \(\circ\) denotes the Hadamard product.
\item
  All dispersion parameters can be modelled as in Eq.(\ref{regdisp}) and
  the model remains a linear covariance model.
\end{itemize}
\end{frame}

\hypertarget{estimation-and-inference}{%
\section{Estimation and Inference}\label{estimation-and-inference}}

\begin{frame}{Estimation and Inference}
\protect\hypertarget{estimation-and-inference-1}{}
\begin{itemize}
\tightlist
\item
  Estimation and inference is carried out using an estimating function
  approach.
\item
  Fitting procedure adapted from Bonat and J\o rgensen, 2016.
\item
  Quasi-score estimating functions for regression parameters.
\item
  Pearson estimating functions for power and dispersion parameters.
\item
  Computational implementation in \texttt{R} through the
  \texttt{mglm4twin}.
\item
  Available on github \texttt{https://github.com/wbonat/mglm4twin}.
\end{itemize}
\end{frame}

\end{document}
