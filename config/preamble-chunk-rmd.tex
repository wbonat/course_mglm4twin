%---- preamble-chunk.tex -----------------------------------------------

% Knitr.

% ATTENTION: this needs `\usepackage{xcolor}'.
\definecolor{color_line}{HTML}{333333}
\definecolor{color_back}{HTML}{DDDDDD}
% \definecolor{color_back}{HTML}{FF0000}

% ATTENTION: usa o fancyvrb.
% https://ctan.math.illinois.edu/macros/latex/contrib/fancyvrb/doc/fancyvrb-doc.pdf
% R input.
\usepackage{tcolorbox}
\ifcsmacro{Highlighting}{
  % Statment if it exists. ------------------
  \DefineVerbatimEnvironment{Highlighting}{Verbatim}{
    % frame=lines,     % Linha superior e inferior.
    % framesep=1ex,    % Distância da linha para o texto.
    % framerule=0.5pt, % Espessura da linha.
    % rulecolor=\color{color_line},
    % numbers=right,
    fontsize=\footnotesize, % Tamanho da fonte.
    baselinestretch=0.9,   % Espaçamento entre linhas.
    commandchars=\\\{\}}
  % Margens do ambiente `Shaded'.
  % \fvset{listparameters={\setlength{\topsep}{-1em}}}
  % \renewenvironment{Shaded}{\vspace{-1ex}}{\vspace{-2ex}}
  \renewenvironment{Shaded}{
    \vspace{2pt}
    \begin{tcolorbox}[
      boxrule=0pt,      % Espessura do contorno.
      colframe=gray!10, % Cor do contorno.
      colback=gray!10,  % Cor de fundo da caixa.
      % arc=1em,          % Raio para contornos arredondados.
      sharp corners,
      % boxsep=0.5em,     % Margem interna.
      left=3pt, right=3pt, top=3pt, bottom=3pt, % Margens internas.
      % grow to left by=0mm,
      grow to right by=6pt,
      ]
    }{
    \end{tcolorbox}
    \vspace{-3pt}
    }
  }{
  % Statment if it not exists. --------------
}

% R output e todo `verbatim'.
\makeatletter
\def\verbatim@font{\linespread{0.9}\normalfont\ttfamily\footnotesize}
\makeatother

% Cor de fundo e margens do `verbatim'.
\let\oldv\verbatim
\let\oldendv\endverbatim

% \def\verbatim{%
%   \par\setbox0\vbox\bgroup % Abre grupo.
%   \vspace{-5px}            % Reduz margem superior.
%   \oldv                    % Chama abertura do verbatim.
% }
% \def\endverbatim{%
%   \oldendv                 % Chama encerramento do verbatim.
%   % \vspace{0cm}           % Controla margem inferior.
%   \egroup%\fboxsep5px      % Fecha grupo.
%   \noindent{\colorbox{color_back}{\usebox0}}\par
% }

%-----------------------------------------------------------------------
